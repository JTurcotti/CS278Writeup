\documentclass{article}

\title{Final Writeup - The Unique Games Conjecture}
\author{Robert Wang and Joshua Turcotti}

\usepackage{amsmath,amsthm,textcomp,amssymb,geometry,graphicx,enumerate}

\newtheorem{conjecture}{Conjecture}
\newtheorem{theorem}{Theroem}

\newcommand{\NP}{\ensuremath{\mathbf{NP}}}
\newcommand{\PCP}{\ensuremath{\mathbf{PCP}}}
\newcommand{\APX}{\ensuremath{\mathbf{APX}}}
\newcommand{\X}{\mathcal{X}}
\newcommand{\Y}{\mathcal{Y}}
\newcommand{\C}{\ensuremath{\mathcal{C}}}
\renewcommand{\a}{\alpha}
\renewcommand{\b}{\beta}
\newcommand{\g}{\gamma}
\renewcommand{\d}{\delta}
\newcommand{\e}{\epsilon}
\newcommand{\I}{\mathcal{I}}
\newcommand{\Ind}{\mathbb{I}}
\renewcommand{\L}{\ensuremath{\mathcal{L}}}
\newcommand{\OPT}{\ensuremath{\mathit{OPT}}}

\begin{document}
\maketitle
\section{Introduction}

This writeup will present our reading and investigations into the Unique Games Conjecture - a robust step in proving innapproximability of certain hard problems. We will formulate it in terms of 2 models - 2 Prover 1 Round Games and Label Cover instances, and justify its usefulness as claimed.

\section{2 Prover 1 Round Games}


In order to reason about reductions from known \NP-Hard problems, we define a class of them in particular, called \textit{2 Prover 1 Round Games} involving a Verifier and 2 Provers. An instance $\I$ of a 2P1R game is defined as follows.
\begin{enumerate}
\item The Verifier asks questions $(X, Y)$, drawn from some distributions $(\X, \Y)$
\item The first Prover receives question $X$ and responds with a proof $\alpha$ from a set $A$, likewise Prover $B$ responds with proof $\beta$ from $B$
\item The Verifier accepts based on some predicate $V(X, Y, \alpha, \beta)$.
\item The value of the game is the probability over $\X, \Y$ that the Verifier accepts given maximal strategies by the Provers.
\end{enumerate}
A 2 Prover 1 Round Game is called \textit{unique} if $A = B$ and the predicate $V$ consists of bijections $\pi_{XY}: A \to B$ for each possible $X, Y$ that must be satisfied by the chosen labels $\a, \b$.



A 2 Prover 1 Round instance $\I$ can be repeated in parallel $n$ times to yield the instance $\I^n$, in which $n$ independent queries $X, Y$ are generated and sent to $n$ non-communicating pairs of Provers. The value of $\I^n$ is the maximum probability that the the Verifier accepts in all $n$ branches. This allows us to formulate the key result \textit{Raz's Parallel Repetition Theorem}:
\begin{theorem}[Raz's Parallel Repetition]
  If $\I$ is a 2P1R instance with value $1 - \e$, then there is a universal constant $\g > 0$ such that the value of $\I^n$ is at most $(1 - \e)^{\g \e^2 n/c}$, where $c$ is a bound on the length of the Prover's answers in $\I$.
\end{theorem}


\section{Label Cover Instances}

We now provide a slight specification of the 2 Prover 1 Round model called the \textit{Label Cover} problem. An instance \L of the Label Cover problem consists of:
\begin{itemize}
\item A complete bipartite graph $G(V, W)$ with bipartition $V, W$
\item A weight $p_{vw}$ assigned to every edge with $\sum_{v, w}p_{vw} = 1$
\item Label sets $N, M$ for vertices in $V, W$ respectively.
\item Projection functions $\pi_{vw} : M \to N$ for every edge $(v, w)$ in $G$
\end{itemize}
An assignment to this CSP consists of choosing label functions $L_V: V \to N$ and $L_W: W \to M$, yielding the following definition:
\[OPT(\L) := \max_{L_V, L_W} \sum_{(v, w) \in G}p_{vw}\cdot\Ind\{\pi_{vw}(L_W(w)) = L_V(v)\}\]
A Label Cover instance $\L$ is called \textit{unique} if $M = N$ and every function $\pi_{vw}$ is a bijection.


Since Label Cover is a CSP, we can apply the PCP theorem to any label cover problem to obtain the \NP-Hardness of a $(\rho, 1)$ Gap version of some instance $\L$ of that problem, for some $\rho$. Since Label Cover is equivalent to a 2 Prover 1 Round game in which the Verifier asks questions from $V, W$ with joint probabilities $p_{vw}$, the provers respond with answers $L_V(v)$ and $L_W(w)$, and the Verifier accepts if the projection functions $\pi_{vw}$ are satisfied, we can apply Raz's Parallel Repetition Theorem to strengthen the Gap to an arbitrarily chosen $\rho$, noting that Label cover just as easy simulates parallel repetition of a 2P1R instance as the original.
\begin{theorem}
  For every constant $\e > 0$, there exists a constant $k = k(\e)$ such that it is \NP-Hard to determine whether a Label Cover instance $\L$ with answers from sets of size at most $k$ (i.e. $|M|, |N| \le k$) has $\OPT(\L) = 1$ or $\OPT(\L) \le \e$.
\end{theorem}

\section{Introducing the Unique Games Conjecture}




The preceeding theorem, $\NP$-Hardness of the $(\e, 1)$ Gap version of Label Cover for arbitrary $\e > 0$ is powerful as the source of reductions to other Gap problems that we wish to show are $\NP$-Hard. However, the requirement of \textit{Perfect Completeness}, i.e. acceptance only if $\OPT(\L) = 1$ is too strong for many desired reductions, so at the cost of restriction to \textit{unique} instances of Label Cover, which can be seen to be equivalent to \textit{unique} instance of 2P1R games, we can relax to imperfect completeness and obtain the following conjecture:
\begin{conjecture}[Unique Games Conjecture]
  For arbitrarily small constants $\e, \d > 0$, there exists a constant $k = k(\e, \d)$ such that it is \NP-Hard to determine whether a unique Label Cover instance $\L$ with label sets of size at most $k$ (i.e. $|M| \le k$) has $\OPT(\L) \ge 1 - \d$ or $\OPT(\L) \le \e$.
\end{conjecture}

Although the UGC is open, is had already been proven to provide powerful inapproximability results, some of which we summarize in figure \ref{ugctable}

\begin{table}[]
  \centering
  \begin{tabular}{|p{2.5cm}|p{2.5cm}|p{2cm}|p{2.5cm}|}
    \hline 
    \textbf{Problem} & \textbf{Best Known Approx.} & \textbf{Best UGC Innapprox.} & \textbf{Best Other Innapprox.} \\\hline
    Vertex Cover & 2 & $2 - \e$ & 1.36 \\\hline
    MaxCut & $\a_{MC} \approx 1.13$ & $\a_{MC} - \e$ & $(\frac{17}{16} \approx 1.06) - \e$ \\\hline
    Max-2SAT &$\a_{LLZ} \approx 1.06$ & $\a_{LLZ} - \e$ & \APX-Hard \\\hline
    Max Acylic Subgraph & 2 & $ 2 - \e$ & $(\frac{66}{65} \approx 1.02)- \e$ \\\hline
    Any CSP $\mathcal{C}$ with integrality gap $\a_\C$ & $\a_\C$ & $\a_\C - \e$ & None \\\hline
    Sparsest Cut & $O(\sqrt{\log{n}})$ & $\omega(1)$ & Hard for some constant factor\\\hline
  \end{tabular}
  \caption{Effect of UGC on innapproximability results}
  \label{ugctable}
\end{table}

We now proceed to elaborate upon such reductions, giving both a specific example and a general framework.


\end{document}
